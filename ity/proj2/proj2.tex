\documentclass[twocolumn, 11pt, a4paper]{article}
\usepackage[utf8]{inputenc}
\usepackage[left=1.5cm,top=2.5cm,text={18cm,25cm}]{geometry}
\usepackage[czech]{babel}
\usepackage[IL2]{fontenc}
\usepackage{amsthm}
\usepackage{times}
\usepackage{amsfonts} 
\usepackage{amsmath}

\theoremstyle{definition}
\newtheorem{definice}{Definice}

\theoremstyle{plain}
\newtheorem{veta}{Věta}

\begin{document}
\begin{titlepage}
\begin{center}
    \Huge
    \textsc{Fakulta informačních technologií}\\
    \textsc{Vysoké učení technické v Brně}\\
    \vspace{\stretch{0.382}}
    \LARGE
    Typografie a publikování -- 2. projekt\\
    Sazba dokumentů a matematických výrazů
    \vspace{\stretch{0.618}}
\end{center}
{\Large 2020 \hfill Jakub Hlava (xhlava52)}
\end{titlepage}
\newpage

\section*{Úvod}
V této úloze si vyzkoušíme sazbu titulní strany, matematických vzorců, prostředí a~dalších textových struktur obvyklých pro technicky zaměřené texty (například rovnice (\ref{rovnice-y}) nebo Definice \ref{def-jazyk} na straně \pageref{def-jazyk}). Pro vytvoření těchto odkazů používáme příkazy \verb|\label|, \verb|\ref| a~\verb|\pageref|.
\par
Na titulní straně je využito sázení nadpisu podle optického středu s~využitím zlatého řezu. Tento postup byl
probírán na přednášce. Dále je použito odřádkování se zadanou relativní velikostí 0.4em a~0.3em.

\section{Matematický text}
Nejprve se podíváme na sázení matematických symbolů a~výrazů v~plynulém textu včetně sazby definic a~vět s~využitím balíku \texttt{amsthm}. Rovněž použijeme poznámku pod čarou s~použitím příkazu \verb|\footnote|. Někdy je vhodné použít konstrukci \verb|${}$| nebo \verb|\mbox{}| která říká, že (matematický) text nemá být zalomen. V~následující definici je nastavena mezera mezi jednotlivými položkami \verb|\item| na 0.05em.
\begin{definice}
\label{ts-definice}
Turingův stroj \emph{(TS) je definován jako šestice tvaru M = ($Q, \Sigma, \Gamma, \Delta, \delta, q_0, q_F$), kde:}
\end{definice}
\begin{itemize} \itemsep0.05em
    \item $Q$ \emph{je konečná množina} vnitřních (řídících) stavů,
    \item $\Sigma$ \emph{je konečná množina symbolů nazývaná} vstupní abeceda, $\Delta \notin \Sigma$,
    \item $\Gamma$ \emph{je konečná množina symbolů}, $\Sigma \subset \Gamma, \Delta \in \Gamma$, \emph{nazývaná} pásková abeceda,
    \item $(Q \backslash \{ q_F \} \times \Gamma \rightarrow Q \times (\Gamma \cup \{L, R\}))$, \emph{kde} $L, R \notin \Gamma$, \emph{je parciální} přechodová funkce, \emph{a}
    \item $q_0 \in Q$ \emph{je} počáteční stav \emph{a $q_f \in Q$ je} koncový stav.
\end{itemize}
\par
 Symbol $\Delta$ značí tzv. \emph{blank} (prázdný symbol), který se vyskytuje na místech pásky, která nebyla ještě použita.
\par
\emph{Konfigurace pásky} se skládá z~nekonečného řetězce, který reprezentuje obsah pásky a~pozice hlavy na tomto řetězci. Jedná se o prvek množiny $\{\gamma\Delta^{\omega}\:|\: \gamma \in \Gamma^*\} \times \mathbb{N}$\footnote{Pro libovolnou abecedu $\Sigma$ je $\Sigma^\omega$ množina všech \emph{nekonečných} řetězců nad $\Sigma$, tj. nekonečných posloupností symbolů ze $\Sigma$}. \emph{Konfiguraci pásky} obvykle zapisujeme jako $\Delta xyz\underline{z}x \Delta$... (podtržení značí pozici hlavy). \emph{Konfigurace stroje} je pak dána stavem řízení a~konfigurací pásky. Formálně se jedná o~prvek množiny $Q \times \{ \gamma \Delta^{\omega}\:|\: \gamma \in \Gamma^*\} \times \mathbb{N}$.
\par
\subsection{Podsekce obsahující větu a odkaz}
\begin{definice}
\label{def-jazyk}
Řetězec $w$ nad abecedou $\Sigma$ je přijat TS \emph{$M$ jestliže $M$ při aktivaci z~počáteční konfigurace pásky} $\underline \Delta w \Delta ...$ \emph{a~počátečního stavu $q_0$ zastaví přechodem do koncového stavu $q_F$ tj. $(q_0, \Delta w \Delta^{\omega}, 0) \overset{*}{\underset{M}{\vdash}} (q_F, \gamma, n)$ pro nějaké $\gamma \in \Gamma^*$ a~$n \in \mathbb{N}$}.
\par
\emph{Množinu $L(M) = \{w\:|\:w$ je přijat TS $M$\} $\subseteq \Sigma^*$ nazýváme} jazyk přijímaný TS $M$. 
\end{definice}
\par
Nyní si vyzkoušíme sazbu vět a důkazů opět s~použitím balíku \texttt{amsthm}.
\par
\begin{veta}
Třída jazyků, které jsou přijímány TS, odpovídá \emph{rekurzivně vyčíslitelným jazykům.}
\end{veta}
\begin{proof}
V důkaze vyjdeme z~Definice \ref{ts-definice} a \ref{def-jazyk}.
\end{proof}
\section{Rovnice}
Složitější matematické formulace sázíme mimo plynulý text. Lze umístit několik výrazů na jeden řádek, ale pak je třeba tyto vhodně oddělit, například příkazem \verb|\quad|.
\par
$$\begin{array}{ccc}
    \sqrt[i]{x^3_i} & \text{kde } x_i \text{ je \emph{i}-té sudé číslo} & y^{2*y_i}_i \neq y_i^{y_i^{y_i}}
\end{array}$$
\par
V rovnici (\ref{rovnice-x}) jsou využity tři typy závorek s~různou explicitně definovanou velikostí.
\begin{align}
    x & = \bigg\{ \Big( \big[ a+b \big] *c \Big) ^d \oplus 1\bigg\} \label{rovnice-x} \\
    y & = \lim \limits _{n \rightarrow \infty} \frac{sin^2 x + cos^2 x}{\frac{1}{log_{10} x}} \label{rovnice-y}
\end{align} 
\par
V~této větě vidíme, jak vypadá implicitní vysázení limity $\lim _{n \rightarrow \infty} f(n)$ v~normálním odstavci textu. Podobně je to i~s~dalšími symboly jako $\sum_{i=1}^n 2^i$ či $\bigcap_{A\in \mathcal{B}} A$. V~případě vzorců $\lim\limits _{n \rightarrow \infty} f(n)$ a~$\sum\limits _{i=1}^n 2^i$ jsme si vynutili méně úspornou sazbu příkazem \verb|\limits|.
\section{Matice}
Pro sázení matic se velmi často používá prostředí \texttt{array} a~závorky (\verb|\left|, \verb|\right|).
\par
$$\left(\begin{array}{ccc}
a+b & \widehat{\xi + \omega} & \hat{\pi} \\
\vec{\mathbf{a}} & \overset{\longleftrightarrow}{AC} & \beta
\end{array}\right) = 1 \Longleftrightarrow \mathbb{Q} = \mathcal{R}$$
Prostředí \texttt{array} lze úspěšně využít i~jinde.

$$\left(\begin{array}{c}
 n \\ k 
\end{array}\right) = \left\{\begin{array}{cl}
0 & \text{pro } k < 0 \text{ nebo } k > n \\
\frac{n!}{k!(n-k)!} & \text{pro } 0 \leq k \leq n
\end{array} \right.$$

\end{document}
