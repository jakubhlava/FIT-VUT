% romano style for circuitikz
% Do not use LaTeX commands if you want it to be compatible with ConTeXt
% Do not add spurious spaces
\tikzset{romano circuit style/.style={%
\circuitikzbasekey/.cd,%
% Resistors
resistors/scale=0.8,
resistors/thickness=1.0,
%
% Capacitors
capacitors/scale=0.7,
capacitors/thickness=1.0,
%
% Inductors (and transformers)
inductors/scale=0.8,
inductors/thickness=1.0,
%
% Diodes
diodes/scale=0.6,
diodes/thickness=1.0,
%
% Batteries
batteries/scale=1.0,
batteries/thickness=2.0,
%
% Sources (the round and diamond-shaped ones!)
sources/scale=0.9,
sources/thickness=1.0,
csources/scale=0.9,
csources/thickness=1.0,
bipoles/noise sources/fillcolor=dashed,
%
% Instruments
instruments/scale=1.0,
%
% Ground symbols
grounds/scale=1.0,
grounds/thickness=1.0,
%
% Transistors
transistors/scale=1.3,
transistors/thickness=3.0,
tripoles/mos style=arrows,  % it can be "arrows"
tripoles/pmos style=emptycircle, % it can be "fullcircle", "nocircle"
transistors/arrow pos=end, % it can be "end"
%
% Amplifiers
amplifiers/scale=1.0,
amplifiers/thickness=3.0,
%
% Logic ports
logic ports/scale=1.0,
logic ports/thickness=2.0,
logic ports origin=center, % it can be "center" (better)
%
% Switches
bipoles/cuteswitch/thickness=0.5,
%
% Integrated circuits
chips/scale=1.0,
chips/thickness=3.0,
%
% other options for romano style
bipoles/crossing/size=0.4,
% I am not sure I like them...
% monopoles/vcc/arrow={Triangle[width=0.8*\scaledwidth, length=\scaledwidth]},
% monopoles/vee/arrow={Triangle[width=0.8*\scaledwidth, length=\scaledwidth]},
},% end .style
}% end \tikzset
% You can add more commands here
% Do not use LaTeX commands if you want it to be compatible with ConTeXt
% Do not add spurious spaces
\endinput

